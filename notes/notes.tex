\documentclass{article}
\usepackage[utf8]{inputenc}
\usepackage{amsmath}
\usepackage{amsthm}
\usepackage{amsbsy}

\title{\LARGE MACHINE LEARNING by Andrew NG}
\author{Ludovico Bessi}
\date{July 2018}

\begin{document}

\maketitle
\tableofcontents
\section{Week 1}
\subsection{Linear regression}
Suppose we have this data: feet and price of some houses. We want to be able to
predict the price of the next house knowking only the feet. The best way to do
that is using linear regression: $$ h_\theta(x) = \theta_0 + \theta_1 x $$
We want to find $\theta_i$ such that the line behaves well with the data.
This is the same as minimizing the following function, called cost function:
$$ J(\theta_0,\theta_1) = \frac{1}{2n}\sum_{i=1}^{n}(h_\theta(x_i) - y_i)^2$$
We can minimize that function using "Gradient descent".
We need to repeat the following operation until convergence:
$$ \theta_j := \theta_j -\alpha \frac{\partial}{\partial \theta_j}
J(\theta_0,\theta_1)$$

This is the same as saying:
$$ \theta_j := \theta_j - \alpha*\frac{1}{n}\sum_{i=1}^{n}(h_\theta(x^{(i)} - y^{(i)})x_j^{(i)}$$

The intuition behind this is that at the minimum the second term goes to 0 as
the derivative goes to 0. $\alpha$ is called the learnig rate, and it is decided
by the user.
\newpage


\section{Week 2}
\subsection{Multivariate linear regression}
Suppose now that we have multiple information about houses.
Then, we want to find a function like this:
$$ h_\theta(x) = \sum_{i=1}^{n} \theta_i x_i $$
We can vectorize that function, obtaining just:
$$ h_\theta(x) = \theta^T x $$

The gradient descent method does not change. However, there is a way to speed
things up. We can normalize the $\theta_i$ to have them all in the same value
range. It is helpful because we minimize the risk of oscillations between values.
We just need to do the following:
$$ x_i := \frac{x_i - \mu_i}{s_i} $$
Where $\mu_i$ is the average and $s_i$ is the standard deviation.

\subsection{Alternative to Gradient descent: Normal equation}
Gradiend descent is an iterative method, it is good even when we have many features.
However it is still iterative, meaning that in principle we may spend a lot of time
computing if we choose the wrong $\alpha$.

We can then use the normal equation if n is not too big:
$$ \theta = (X^T X )^{-1} X^T y $$
We need to be careful for two reasons: it may be the case that $X^T X$ is not
invertible, because there are linearly dependent features, or too many of them.
Moreover, calculating $\theta$ in this way costs $O(n^3)$.

\section{Week 3}
\subsection{Linear classification}
Suppose we get an e-mail. We want to understand whether or not it is spam.
We need a function that takes as input all the features of said email, and
outputs either 0 or 1. We use the sigmoid function to achieve this.
Let $ g : \mathbf{R} \to (0,1) $ be the function $g(z) = \frac{1}{1+e^{-z}}$.
Then we can write $h_\theta(x) = g(\theta^T x)$.
For example, if we have $h_\theta(x) = 0.7$, this means that we 70\% of times the
output will be 1. We define y = 1 if $h_\theta(x) \geq 0.5$ and y = 0 otherwise.

\subsection{Logistic regression model}
It is evident that we cannot use the same cost function as the regression model.
Instead, we need to use the following:
$$ J(\theta) = \frac{1}{m}\sum_{i=1}^m Cost(h_\theta(x^{(i)}),y^{(i)}) $$,
with $Cost(h_\theta(x^{(i)}),y^{(i)}) =
-ylog(h_\theta(x)) - (1-y)log(1-h_\theta(x))$
We then have the vectorized equation, with $h = g(X\theta)$:
$$ J(\theta) = \frac{1}{m}(-y^Tlog(h)-(1-y)^Tlog(1-h)) $$

\subsection{Multiclass classification}
We now treat the case with $y \in \{0,1,2,...,n\}.$
We divide the problem in n+1 sub problems of binary classification, the prediction
will be the maximum value of $h_\theta(x)$.

\subsection{Overfitting}
If we try to fit too many features, we end up with a function that cannot predict
well new results. To avoid it, we need to choose which features to analyze or to
reduce the magnitude of the parameters.
Let's see this last point in more detail: we need to change the cost function to
avoid overfitting.
Suppose we have:
$$ \theta_0 + \theta_1x + \theta_2x^2  + \theta_3x^3 + \theta_4x^4 $$
We do not want to ditch the last terms, but we want to reduce their influence.
The new cost function will then be like this:
$$ min_{\theta}\frac{1}{2m}\sum_{i=1}^{m}(h_\theta(x^{(i)})-y^{(i)})^2 +
1000 \theta_3^2 + 1000 \theta_4^2 $$
When the cost function goes to 0, inevitably the terms $\theta_3$ and $\theta_4$
must go to 0 as well.
We can rewrite it with $\lambda$ as regularization parameter:
$$ min_{\theta}\frac{1}{2m}\sum_{i=1}^{m}(h_\theta(x^{(i)})-y^{(i)})^2 + \lambda
\sum_{j=1}^n\theta_j^2$$

We must be careful with the choiche of $\lambda$, if it is too large we could
risk underfitting, moreover the Gradient descent method won't converge.
If it is too little, we do not remove overfitting.

\subsection{Regularized linear regression}

$$\theta_j := \theta_j(1-\alpha\frac{\lambda}{m})-
\frac{\lambda}{m}\sum_{i=1}^m(h_\theta(x^{(i)}) - y^{(i)})x_j^{(i)} $$

$$ \theta = (X^T X + \lambda \cdot L)^{-1} X^T y $$

\subsection{Regularized logistic regression}

$$ J(\theta) = -\frac{1}{m}\sum_{i=1}^m[y^{(i)}log(h_\theta(x^{(i)})) +
(1-y^{(i)})]log(1-h_\theta(x^{(i)}))] + \frac{\lambda}{2m}\sum_{j=1}^n\theta_j^2$$


\section{Week 4}
\subsection{Introduction}
In this section the main topic is neural networks. The focus is on understanding
how to build them, with the help of some examples to build intuition.
\subsection{Model representation}
The input is a vector $x_i$ of features. $x_0$ is sometimes called the bias
feature. We can still use the sigmoid activation function
$\frac{1}{1 - e^{\theta^T x}}$. Between the input and the output, there may be
more layers, called hidden layers. The layers are "connected" using weights.
Suppose we have just one hidden layer. We can find the first value of that
vector as following:
$$ a_1 = g(\theta_{10}^{(1)}x_0 + \theta_{11}^{(1)}x_1 + \theta_{12}^{(1)}x_2 +
          + \theta_{13}^{(1)}x_3)$$
Let's explain it a bit. To get the \textbf{first} value of the layer, we
just multiplied the \textbf{first} row of the matrices of weigths by the features.
Osservaton: if layer 1 has 2 input nodes and layer 2 has 4 activation nodes, then our matrix
$\theta_{ij}$ has dimension 4x3.
\subsection{Examples and intuitions}
\subsubsection{AND operator, OR operator}
Recall that a AND b == TRUE if and only if both a and b are true.
Then we construct a neural network as follows:
$$\begin{pmatrix}x_0\\x_1\\x_2\end{pmatrix} \to
  \begin{pmatrix}g(z^{(2)})\end{pmatrix} \to
  h_\theta(x)$$
With $\theta_{1,3} = [-30 20 20]$, then our output will be positive if both
$x_1$ and $x_2$ are 1. Remember that $x_0$ is always fixed at 1.

By setting $\theta_{1,3} = [-10,20,20]$ we get the OR operator.

\subsection{XNOR operator}
By combining OR, AND and XOR we can get the XNOR operator. XNOR will give 1 if
$x_1$ and $x_2$ are both 0 or 1. To do so, we need to
add another hidden layer:

$$\begin{pmatrix}x_0\\x_1\\x_2\end{pmatrix} \to
  \begin{pmatrix}a_1^{(2)}\\a_2^{(2)}\end{pmatrix} \to
  \begin{pmatrix}a_1^{(2)}\\a^{(3)}\end{pmatrix} \to
  h_\theta(x)$$

For the transition from the input layer to the first hidden layer, we use a matrix
that combines AND and NOR:
$$ \theta^{(1)} = \begin{pmatrix} -30\ 20\ 20\\ 10 -20 -20\end{pmatrix} $$

For the transition between second and third layer, we use the values for OR:

$$ \theta^{(2)} = [-10,20,20]$$

\subsection{Multiclass classification}

The output of the neural network may as well be a vector and not just a single
number. Suppose we have a photo of a vehicle and we need to tell if it is a car,
a truck or a scooter. The output vector will have three numbers ranging from 0
to 1, that indicate the level of confidence of our claim.

\section{Week 5}
The main topic will be: How to train a Neural network?
\subsection{Cost function}
The cost function is similar to the logistic regression cost function, we just
need to add a summation in the first term, because it needs to loop through all
the output nodes. In the regularization term, we need to add two summations,
because we must take into account multiple $\theta$ matrices.
\subsection{Back propagation algorithm}
Backpropagation means: "minimizing cost function".
\subsection{Error in back propagation}
We call $\delta$ the error in the backpropagation algorithm. It is calculated by
multiplying the following $a_i$ nodes by the corresponding $\theta_{i,j}$.

\subsection{Gradient checking}
The following formula is useful when it is needed to check whether or not the
backpropagation algorithm is working fine:

$$\frac{\partial}{\partial \theta}J(\theta) \approx \frac{J(\theta + \epsilon)
- J(\theta - \epsilon)}{2\epsilon} $$

\subsection{random initialization}
Setting all theta's to 0 at the start will not work. It is needed to set those
in a random range $[-\epsilon,+\epsilon]$.
\subsection{Wrapping it up}
The following steps are required to train a neural network:
\begin{itemize}
  \item Randomly initialize the weights
  \item Implement forward propagation to get $h_\theta(x)$ for all $x_i$
  \item Implement Cost function
  \item Using back propagation to compute partial derivatives
  \item Use gradient checking one time to be sure that back propagation
        is working properly.
  \item Use gradient descent or other built-in functions to minimize the cost
        function with the weights.


\end{itemize}

\section{Week 6}
The main topic is about understanding if a given neural network works properly.
Moreover, How can we speed up the algorithm?

\subsection{Evaluating hypothesis}
An idea is to split our data: 70\% will be training set and the remaining 30\%
will be the set test. Then we introduce a test error function:
$$err(h_\theta(x),y) = 1 $$ if $h_\theta(x) \geq 0.5 and  y = 0$ or the opposite.

\subsection{Bias vs Varince}
High bias $=$ underfitting while High variance means overfitting. \\

High bias: $J_{train}$ and $J_{CV}$ very high. \\
High variance: $J_{CV} >> J_{train}$.

Having more data helps with high variance, while it does not help if we have
high bias.

More training examples, smaller set of features and increasing $\lambda$ helps
with high variance. \\
Adding features and decreasing $\lambda$ helps with high bias.

\subsection{Week 7}
Support vector machine is one of the most useful black box algorithm for machine
learning. It uses different functions you can choose (kernels) that are handy
when tracting similarity. Useful to perform feature scaling when using Gaussian
kernel. n = # of featues, m = # of training examples.

if n $geq m$, use logistic regression or SVM without kernel.
If n is small and m intermediate, use SVM with Gaussian kernel.
If n is small and m is large, add more features then use logistic regression or
SVM without kernel.

\subsection{Week 8}
The main topics will be clustering and principal component analysis, in the
setting of unsupervised learning.

\subsubsection{Clustering and K-mean algorithm}
We have some data, and we need to classify it. To do so, we pick K centroids and
we run K-mean algorithm to optimise their position. That is, to minimize the
total distance between the $K_i$ centroid and the cluster points. At first,
we need to randomize the centroids. It is higly inneficient to just choose random
points. Instead, we should pick as starting centroid a point of the cluster.

\subsubsection{Data compression and Principal component analysis}

Data compression means moving your data set from nD to mD, with m $<$ n.
(Example: a line in $\mathbb{R}^2$ becomes point on the real line).
Principal components analysis is deeply interconnected with data compression:
if we have a scatter plot, then going to 1D may be tricky.
We neeed to project in such a way to reduce the error.
We do that by computing the covariance matrix, computing the eigenvectors and we
multiply U by x, where U is [U,S,V] = svd(covariance). You pick k such that
99\% of covariance is retained.
PCA is not useful if you want to avoid overfitting, in that case it is best to
use normalisation.



\subsection{Week 9}



\subsection{Week 10}


\subsection{Week 11}
\end{document}
